\chapter{Conclusion}

\begin{flushleft}
We began this project to explore a seemingly simple conjecture. Why would it be the case that a minimally rigid graph would not give rise to a infinitesimally rigid circle packing? The technqiues involved to research this question computationally was a lot more work than one may have anticipated however.
\end{flushleft}

\begin{flushleft}
Recapping the journey we have undertaken, we first learned about what frameworks are and how it differs from a graph. Graphs are abstract mathematical structures, whereas each node in a framework is equipped with its own position vector in $\mathbb{R}^d$. We then defined rigidity, and what it meant for a framework to be rigid in $\mathbb{R}^d$. A stronger property of rigidity, infinitesimal rigidity, was brought up, stating that the the framework should not deform when velocity vectors are applied to any node in the framework. This allowed us to note that for a framework to be rigid, the only way it is permitted to move is either by rotating it, or translating it through space.
\end{flushleft}

\begin{flushleft}
Furthering our study of rigidity, we built up a collection of theorems in order to deduce whether a given framework $(G,\textbf{p})$ is infinitesimally rigid or not, including a way to encode a framework into a matrix. Realizing any infinitesimally rigid framework must have 3 degrees of freedom, as well as noticing that the kernel of the rigidity matrix was simply the space of infinitesimal rigid motions allowed us to quickly deduce that the rank of the rigidity matrix must be $2n-3$ with the use of the Rank-Nullity Theorem, where $n = |V(G)|$. 
\end{flushleft}

\begin{flushleft}
Next, we introduced the concept of circle packings, and defined the contact graph of a circle packing. Studying the contact graph allowed us to use results developed earlier for graphs and frameworks to study circle packings. The Circle Packing Theorem was seen here, and played a major role when finding the circle packings of interest. 
\end{flushleft}

\begin{flushleft}
Finally, we built two modules in Python named \texttt{Rigidity.py} and \texttt{Circle\_Packing.py}. Leveraging them to test a number of planar graphs for rigidity, and then attempt to find circle packings for each of them, we saw that all except one graph on $n$ vertices, where $n \in [3, \hdots, 9]$, satisfied the conjecture by this numerical method. However, the code developed here was limited in the scenarios it could tackle, and so not all the minimally rigid graphs on $n = 10$ nodes could be studied in a time and energy efficient manner. 
\end{flushleft}

\begin{flushleft}
In the end, we were able to explore the conjecture defined at the start of this project in quite some detail. The goal was to gain some insight into the truth of the conjecture, which we now have! Knowing that it is true for all minimally rigid graphs up to 9 nodes, and true for nearly a third of all minimally rigid graphs on 10 nodes was an exciting adventure in and of itself. Perhaps with some modifications to the existing code, by running the code in multiple environments, and the use of analytical techniques like those involved in studying the rigidity of circle packings, we may be able to explore a larger number of nodes. 
\end{flushleft}

\begin{flushleft}
As mentioned at the start of Chapter 4, the last part of this project was similar to the final boss fight in a video game. The theorems involving rigidity formed our arsenal of tools, with which we crafted two formidable weapons, the modules \texttt{Rigidity.py} and \texttt{Circle\_Packing.py}. By engaging this fearsome foe head-on, we were able to learn what methods of attack work against this enemy, and became aware of areas that may need improvement. To beat this boss though, we will need to deal with the limitations faced earlier, and possibly learn new skills to approach this fight at a different angle.
\end{flushleft}

\begin{flushleft}
However, that is a journey we may wish to embark on at a later time. For now, this is a good point to save our progress, quit the game and get some well-deserved rest. 
\end{flushleft}